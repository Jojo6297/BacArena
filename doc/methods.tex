\subsection{Model overview}

\begin{algorithm}
\SetAlgoLined
\For{\emph{time} in $1:$\emph{iterations}}{
  \emph{diffusion}()\;
  \For{\emph{l} in $1:$\emph{BacNumbers}}{
    \emph{fba}()\;
    \emph{movement}()\;
    \emph{growth}()\;
  }
}
\caption{Main loop called by \texttt{diffbac.R}}
\end{algorithm}

\subsection{Representation}
Implementation began in \textit{netlogo}, which is a simple and wide-used agent based modeling framework.\cite{Wilensky1999}
For statistical analysis the powerful \textit{R} language was our choice. 
There exists joint package for interaction of \textit{R} and netlogo (e.g.: \cite{Thiele2010}), but above all speed limitations led us looking for other possibilities.
In \textit{R} there exists packages to do agent based modeling e.g. \textit{simecol}, which is an ecological framework with differential equation based approach, too. \cite{Petzoldt2007}.
But again we could retain performance improvement by simply implementing it ourselves directly in \textit{R}.

\subsubsection{Environment \& Grid}
Agents, which are called \textit{Bacs} in BacArena, exists on a \textit{grid} representation.
A grid is a discretization of space and could be imagined as a chess board, where Bacs move like chess pieces. 

\subsubsection{Bacteria}
\subsubsection{Substrate}

\subsection{Interactions as rules}
\subsubsection{Movement}
\subsubsection{Diffusion}
\subsubsection{Flux balance analysis}

\subsection{Growth}
